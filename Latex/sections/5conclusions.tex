\section{Conclusions}
\label{sec:conclusion}

%%% PROTOYPE VS EXEMPLARS

% Shown how prototype and exemplar effects control the alternation.

% Unnecessary to follow extreme routes for no-abstraction or full-abstraction model.

% Show how multilevel modelling helps to specify complex interactions between abstract features, item-specific tendencies and exemplar effects.

% All in all, clear evidence for probabilistic nature of the phenomenon.

%%% CONVERGENCE AND NON-CONVERGENCE

% There are now a number of published multivariate models that use data, extracted from corpora and annotated for a multitude of morphological, syntac- tic, semantic and pragmatic parameters, to predict the choice for one morpheme, lexeme or construction over another. However, most of these studies are con- cerned with phenomena that involve binary choices (Gries 2003; De Sutter et al. 2008) and only a small number of these1 corpus-based studies have been cross-validated (Keller 2000; Wasow and Arnold 2003; Sorace and Keller 2005; Roland et al. 2006; Arppe and Järvikivi 2007; Divjak and Gries 2008).2 Of these cross- validated studies, few have directly evaluated the prediction accuracy of a complex, multivariate corpus-based model on humans using authentic corpus sentences (with the exception of Bresnan 2007; Bresnan and Ford 2010; Ford et al. 2013a; Ford et al. 2013b), and even fewer have attempted to evaluate the prediction accuracy of a polytomous corpus-based model in this way (but see Arppe and Abdulrahim 2013 for a first attempt). Below we will review the latter two types of cross-validated studies.

% Pre-study and main study mostly converge, although Badness is weak in pre-study. Makes it highly likely that the postulated factors are substantial.
% Also: Genitives is a good measure, corroborated through pre-study.

% Notice that overall fit of main study is good R2m>0.4 but not anywhere near perfect predictability.
% Due to more subtle preferences and more free variation. Rarely do speakers feel one alternative is odd or bad. => More experiments.

% Important: Nuisance factor (case) in line with Zimmer2015, which improves the validity of the presented results.

% Weak convergence with experiments: Consider lower R2 in main study as level to begin with. No wonder that experimental corroboration is even weaker.
% See also extreme inter-speaker variation, and possibly problematic stimuli which marred results. => More experiments.
% Why self-paced reading better? Less room for normative evaluations, and genitives are a hot matter of public debate.

% (External) things that were NOT corroborated:
% Collexeme strength failed against raw frequency-based measure.
% GAM did not improve fit in reading time experiment. Effect was quite linear.
% Overparametrisation was ill-advised as predicted by BatesEa2015a, MatuschekEa2017.

% Mention pre-registration.

% We need: more case studies +  diverse and different corpora + focus on register/style-specific aspects + inter-speaker similarity + focus on different experimental paradigms and what they really tell us + more clarity wrt statistical methods + ultimately fully specified mathematical models as in cognitive science

Besides the methodological aspects mentioned above, future work could extend the results and provide a general picture of the constructions expressing measurements (see Section~\ref{sec:descriptive}).
This would be a much more complicated task given that the choices then would no longer be binary and that the meaning of the alternative ways of expressing measurements are semantically more varied.

% Potential refinement: more detailed analysis instead of CARD vs. not CARD
% Potential refinement: postnominal modification (= true partitives)
% Regional variation (experiments)
% Mention R1 and his ``lexicalised A+N combinations''

Finally, I want to point out that German is mildly under-researched in the specific framework used here.
This is quite surprising given the fact that German morpho-syntax is famous for its alternations, which are usually called \textit{Zweifelsfälle} (`cases of doubt') in the traditional literature \citep{Duden09,Klein2009}.
Instead of being drowned in normative, descriptive, or didactic discussions, they could serve as ideal test cases in cognitive linguistics.

