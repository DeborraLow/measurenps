\section{Conclusions}
\label{sec:conclusion}

\hl{Mention pre-registration.}

\hl{The fact that the correlation is not stronger can be explained in parts \ldots}

\hl{Wen need: more case studies +  diverse and different corpora + focus on register/style-specific aspects + inter-speaker similarity + focus on different experimental paradigms and what they really tell us + more clarity wrt statistical methods + ultimately fully specified mathematical models as in cognitive science}

\hl{Potential refinement: more detailed analysis instead of CARD vs. not CARD}

\hl{Potential refinement: postnominal modification (= true partitives)}

This paper stands in a now ten year-old tradition of research on grammatical alternations using corpus and experimental data.
In my view, the main tenets of this line of research are:
(i)~Language, viewed from a cognitively realistic angle, is a probabilistic phenomenon and cannot be modelled appropriately within Aristotelian frameworks that assume discrete categories.
(ii)~Corpora are collections of usage events (language production) and can therefore be used to evaluate both the claim made in (i) and specific theoretical claims (in case studies) about factors influencing speakers' decisions to use specific forms or constructions.
(iii)~Given (ii), we expect results from corpus analyses and from appropriate experiments to yield similar results, not necessarily as a form of validation of the corpus-based findings, but as converging evidence.
I consider these three points to be of utmost importance because they clearly set this approach to linguistic research apart from \textit{both} Aristotelian frameworks \textit{and} introspective, non-empirical, and anti-quantitative versions of cognitive linguistics (see \citealp{Dabrowska2016} for a pithy philippic against such approaches).

The present paper adds to the evidence that all of the aforementioned three points are correct.
A grammatical alternation in German measure NPs was examined using corpus data based on factors partly derived from existing accounts, formulated in terms of construction prototypes.
The preferences extracted from the DECOW web corpus were confirmed in a forced-choice experiment, in which participants explicitly chose alternants in line with the probabilities derived from the corpus-based model.
In a more implicit self-paced reading experiment, it was shown that the much rarer alternant brings about a reading time penalty except in cases for which the corpus model predicts very high probability for this alternant.

Future work could extend these results and provide a general picture of the constructions expressing measurements (see Section~\ref{sec:descriptive}).
This would be a much more complicated task given that the choices then would no longer be binary and that the meaning of the alternative ways of expressing measurements are semantically more varied.
Finally, I want to point out that German is mildly under-researched in the specific framework used here.
This is quite surprising given the fact that German morpho-syntax is famous for its alternations, which are usually called \textit{Zweifelsfälle} (`cases of doubt') in the traditional literature \citep{Duden09,Klein2009}.
Instead of being drowned in normative, descriptive, or didactic discussions, they could serve as ideal test cases in cognitive linguistics.


