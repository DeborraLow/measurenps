
\section{Cognitively oriented corpus linguistics}
\label{sec:cogocl}

This paper deals with a morpho-syntactic alternation between two constructions that occurs only in a very specific type of measure noun phrase in German.
By \textit{alternation} I refer to a situation where two or more forms or constructions are available with no clear difference in acceptability, function, or meaning.
The study of alternations has a long history in cognitively oriented corpus linguistics (\citealp{BresnanEa2007,BresnanHay2010,BresnanFord2010,DivjakArppe2013,Gries2015,NessetJanda2010}, to name just a few publications).
This area of research is based on the assumption that language is a probabilistic phenomenon \citep{Bresnan2007} where alternants are chosen neither deterministically nor fully at random.
Instead, multifactorial models are constructed which incorporate influencing factors from diverse levels, including contextual factors.
The estimation of the model coefficients quantifies the influence that the factors have on the probability that either alternant is chosen.

For such probabilistic generalisations estimated on corpus data, there has always been an interest in correlating the findings with results from experimental work (for example, \citealp{ArppeJaervikivi2007,BresnanEa2007,BresnanFord2010,DivjakGries2008,DivjakEa2016,FordBresnan2013}).
This is often called a \textit{validation} of the corpus-derived findings, but \citet[303]{Divjak2016a} rightly criticises this choice of words ``because it creates the impression that behavioral experimental data is inherently more valuable than textual data,'' citing \cite{TummersEa2005}, who state that a corpus is ``a sample of spontaneous language use that is (generally) realized by native speakers.''
However, as \citet[486--487]{Dabrowska2016} convincingly argues, this does not imply that we can in some way ``deduce mental representations from patterns of use,'' \ie\ from corpus data.%
\footnote{In its colloquial meaning, the verb \textit{deduce} seems appropriate here.
However, in terms of the scientific method, we are most likely talking about \textit{induction} rather than \textit{deduction}.}
It would be highly surprising if this were possible, and the same holds for experimental methods.
Nobody assumes that we can inductively infer mental representations from experiments, which -- as opposed to corpus studies -- even allow for direct access to the cognitive agent and offer much better possibilities to control experimental conditions and nuisance variables.
Rather, under the standard approach, a theory of cognitive representation is pre-specified.
Then, predictions are derived from this theory \textit{before} the experiment or the corpus study is conducted to \textit{test} the theory.
Technical problems of statistical inference aside, a successful experiment corroborates the theory, and a failed experiment weakens the theory.
The simplest assumption would be that the same approach is adequate for corpus studies.

The present study is conducted within the general paradigm of cognitive corpus linguistics and includes a comparison of the corpus findings with results from two experiments.
I assume a model of similarity-based classification in the form of Prototype Theory for modeling alternation.
Protoype Theory has been used in alternation research in cognitively oriented corpus linguistics (see \citealp{DivjakArppe2013}; \citealp{Gries2003}).
It is assumed that the variables annotated in a corpus study on an alternation phenomenon define prototypes for the alternants.
The similarity of a given feature vector (in a concrete sentence where one of the alternants is used) influences to a large extent which alternant is chosen by speakers.
This entails that a variant of Prototype Theory with features is assumed \citep{Rosch1978}, instead of the monolithic prototypes of earlier versions.
Prototype Theory is well suited for modeling constructional choices but it is just one of several similarity-based theories of classification, the most prominent other framework being Exemplar Theory (\citealp{MedinSchaffer1978,Hintzman1986}; see \citealp{StormsEa2000} for a comparison of the theories in experimental settings).
Prototype Theory and Exemplar Theory model essentially the same types of effects and differ mainly in whether they assume higher-level abstractions as part of the cognitive representation (Prototype Theory) or try to do without them (Exemplar Theory).
I am sceptical that corpus analysis alone could ever decide which theory is more suitable, given that substantial doubt has been voiced whether even experimental methods are ultimately able to do so \citep{Barsalou1990}.%
\footnote{In \cite{DivjakArppe2013}, it was shown using corpus data how both models form a converging picture.}
Prototype Theory (with feature abstractions) is preferred here simply because it fits the established alternation modeling paradigm, which relies on features being weighted in statistical models.
In Section~\ref{sec:analyses}, a prototype-theoretical parlance is therefore adopted.

In the remainder of the paper, I introduce the alternation phenomenon (Section~\ref{sec:descriptive}) and suggest a theory-driven set of factors influencing the alternation (Section~\ref{sec:analyses}).
Then the corpus study is presented, including an appropriate statistical analysis (Section~\ref{sec:corpusstudies}).
Two experiments which corroborate the corpus-based findings are then reported in Section~\ref{sec:experimental} before I sum up the paper in Section~\ref{sec:conclusion}.


%%%%%%%%%%%%%%%%%%%%%%%%%%%%%%%%%%%%%%%%%%%%%%%%%%%%%%%%%%%%%%%%%%%%%%%
%%%%%%%%%%%%%%%%%%%%%%%%%%%%%%%%%%%%%%%%%%%%%%%%%%%%%%%%%%%%%%%%%%%%%%%
%%%%%%%%%%%%%%%%%%%%%%%%%%%%%%%%%%%%%%%%%%%%%%%%%%%%%%%%%%%%%%%%%%%%%%%


