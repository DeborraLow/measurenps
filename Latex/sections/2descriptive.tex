
\section{Case assignment in German measure NPs}
\label{sec:germanmeasurenps}


\subsection{Two stable cases and a case alternation}
\label{sec:descriptive}

In this section, I introduce and illustrate the relevant alternating constructions.
I describe the narrowly defined syntactic configuration in which the alternation occurs, and I motivate the focus on \textit{only} this narrow range rather than, for example, the whole range of nominal constructions expressing quantities.

I use the term \textit{measure noun phrase} (MNP) to refer to a noun phrase (NP) in which a kind-denoting (count or mass) noun depends on another noun that specifies a quantity of the objects or the substance denoted by the kind-denoting noun.
I call the kind-denoting noun the \textit{kind noun} and the quantity-denoting noun the \textit{measure noun}.
For illustration purposes, in the English \textit{a glass of good wine}, \textit{glass} is the measure noun and \textit{wine} is the kind noun.
Measure nouns can be all sorts of nouns which denote a quantity (such as \textit{litre} or \textit{amount}) but also those denoting containers, collections, etc. (such as \textit{glass} or \textit{bucket}).
Like \citet[284]{Brems2003}, I consider nouns as measure nouns ``which, strictly speaking, do not designate a `measure', but display a more nebulous potential for quantification'' (also \citealp[530]{Koptjevskaja2001}, and \citealp[338]{Rutkowski2007}).

\subsubsection{Core structures related to the alternation}

Three different syntactic configurations within MNPs need to be distinguished, and the case alternation occurs only in one of them.
It occurs only when the kind noun is modified by an attributive adjective and there is no determiner, as in (\ref{ex:intro:alternation}).
Superficially, the sentences are functionally and semantically equivalent either with the kind noun in the genitive (\ref{ex:intro:alternation1}) or in the same case as the measure noun, an accusative in the case of (\ref{ex:intro:alternation2}).%
\footnote{Some descriptive and normative grammars take stronger positions with regard to the acceptability of the two options.
See \cite{Hentschel1993} and \cite{Zimmer2015} for analyses of the sometimes absurd stances taken in grammars of German.
As will be shown (especially in Section~\ref{sec:exp:fc}), there might be preferences, but we cannot assume either construction to be unacceptable.
}

\begin{exe}
  \ex\label{ex:intro:alternation}
  \begin{xlist}
    \ex[ ]{\label{ex:intro:alternation1} \gll Wir trinken [[ein Glas]\Sub{Acc} [guten Weins]\Sub{Gen}]\Sub{Acc}.\\
    we drink a glass good wine \\
    \glt We drink a glass of good wine.}
    \ex[ ]{\label{ex:intro:alternation2} Wir trinken [[ein Glas]\Sub{Acc} [guten Wein]\Sub{Acc}]\Sub{Acc}.}
  \end{xlist}
\end{exe}

This specific configuration has to be seen in the context of two other configurations, to which I turn now.
First, if the kind noun forms an NP with a determiner, the construction resembles (and is usually called) a \textit{pseudo-partitive} (on partitives and pseudo-partitives see, e.\,g., \citealp{Barker1998,Selkirk1977,Stickney2007,Vos1999}; for a recent application of the terminology to German, see \citealp{Gerstenberger2015}).%
\footnote{If the kind noun is definite, the construction instantiates a true partitive.
Whereas partitives are constructions denoting a proper part-of relation as in \textit{a sip of the wine}, pseudo partitives -- albeit syntactically similar and diachronically related to partitives in many languages -- merely denote quantities and contain indefinite kind nouns as in \textit{a sip of wine}.
In the literature on German, some authors incorrectly call the pseudo-partitive a \textit{partitive} \citep{Hentschel1993} while some realise the difference and at least mention it \citep{Eschenbach1994,GallmannLindauer1994,Loebel1989,Zimmer2015}.
}
Here, the kind noun is in the genitive, and I refer to the construction in (\ref{ex:intro:pseudopartitive1}) as the \textit{Pseudo-partitive Genitive Construction} (PGC).

\begin{exe}
%  \ex\label{ex:intro:pseudopartitive}
%  \begin{xlist}
    \ex[ ]{\label{ex:intro:pseudopartitive1}\gll Wir trinken [[ein Glas]\Sub{Acc} [dieses Weins]\Sub{Gen}]\Sub{Acc}.\\
    we drink a glass this wine\\
    \glt We drink a glass of this wine.}
%    \ex[*]{\label{ex:intro:pseudopartitive2} Wir trinken [[eine Tasse]\Sub{Acc} [einen leckeren Kaffee]\Sub{Acc}]\Sub{Acc}.}
%  \end{xlist}
\end{exe}

Second, if the kind noun is bare -- \ie\ if it comes neither with a determiner nor a modifying adjective -- it is uninflected as in (\ref{ex:intro:narrowapposition1}), and the genitive as seen in the PGC is not acceptable, see (\ref{ex:intro:narrowapposition2}).

\begin{exe}
  \ex\label{ex:intro:narrowapposition}
  \begin{xlist}
    \ex[ ]{\label{ex:intro:narrowapposition1}\gll Wir trinken [[ein Glas]\Sub{Acc} [Wein]\Sub{?}]\Sub{Acc}.\\
    we drink a glass wine\\
    \glt We drink a glass of wine.}
    \ex[*]{\label{ex:intro:narrowapposition2} Wir trinken [[ein Glas]\Sub{Acc} [Weins]\Sub{Gen}]\Sub{Acc}.}
  \end{xlist}
\end{exe}

This construction is usually classified as a \textit{Narrow Apposition Construction} \citep{Loebel1986}, henceforth NAC.%
\footnote{The construction as in (\ref{ex:intro:narrowapposition1}) is also referred to as the \textit{Direct Partitive Construction} for other Germanic languages in which the PGC with the synthetic genitive is not available.
This nomenclature makes sense in contrast to the \textit{Indirect Partitive Construction} with prepositional linkers translating to \textit{of} -- \ie\ analytic genitives -- in such languages, see \cite{HankamerMikkelsen2008} for Danish.
For German, this terminology is not distinctive enough, which is why I use the terms NAC and PGC.}
Notice that the unavailability of the genitive on the kind noun follows independently from a constraint that genitive NPs in German require the presence of some strongly case-marked element (determiner or adjective) in addition to the head noun in order to be acceptable (\citealp{GallmannLindauer1994,Schachtl1989}; see also \citealp[160]{Eisenberg2013b}).

It is difficult to determine whether the bare kind noun in the narrow apposition construction as in (\ref{ex:intro:narrowapposition1}) bears no case at all, a generic case, or agrees in case with the measure noun.%
\footnote{I would like to thank one of the reviewers for pointing this out.}
When there is an adjective as in (\ref{ex:intro:alternation2}), the embedded kind NP clearly agrees in case, but due to the overall absence of markers of case in the singular, bare nouns mostly show no indication of their case.
The only nouns which do have case markers in the singular are the so-called weak nouns \citep{Koepcke1995,Schaefer2016c}, which have something like a non-nominative \textit{-en} marker in the singular.
Unfortunately, there are no genuine mass nouns among the weak nouns.
However, a few of them can be coerced into a mass noun, such as \textit{Hase} `rabbit' (meaning `rabbit meat').
It then appears as if the uninflected form is preferred, but the inflected form is not excluded.
In (\ref{ex:macabre1}), the clearly acceptable form \textit{Hase} can only be a nominative singular or caseless.
In (\ref{ex:macabre2}), the form \textit{Hasen} could be an accusative, dative, or genitive.

\begin{exe}
  \ex\label{ex:macabre}
  \begin{xlist}
    \ex[ ]{\label{ex:macabre1}\gll Niemand will [ein Stück [Hase]\Sub{Nom\slash caseless}]\Sub{Acc} essen.\\
    nobody wants a piece rabbit eat\\
  \trans Nobody wants to eat a piece of rabbit.}
    \ex[?]{\label{ex:macabre2}\gll Niemand will [ein Stück [Hasen]\Sub{Acc\slash Dat\slash Gen}]\Sub{Acc} essen.\\
    nobody wants a piece rabbit eat\\}
  \end{xlist}
\end{exe}

Even a full unacceptability of (\ref{ex:macabre2}) would not be conclusive, however, as a possible aversion of speakers towards the case-marked form might be due to the fact that it is at least potentially also a genitive, in which case the constraint against bare genitive nouns would apply.
With plural kind nouns, the obligatory marking of the dative plural with \textit{-en} might provide some clues.
However, plural kind nouns do not behave like singular ones in measure phrases anyway, as will be argued in Section~\ref{sec:neighbouringcases}.
Also, judgements vary between (\ref{ex:datpl1}) and (\ref{ex:datpl2}), and both are found in corpora.%
\footnote{For example, the variant in (\ref{ex:datpl1}) with the lemmas \textit{Sack} and \textit{Apfel} occurs four times, and the one in (\ref{ex:datpl2}) twice in the very large DECOW corpus (see Section~\ref{sec:gettingdata}).
Similarly, for [\textit{Kiste} [\textit{Äpfel}]\Sub{Nom\slash Acc\slash Gen}]\Sub{Dat} `box of apples' (no case identity), I find three examples, and [\textit{Kiste} [\textit{Äpfeln}]\Sub{Dat}]\Sub{Dat} (clearly marked case identity), I find six examples.
}

\begin{exe}
  \ex\label{ex:datpl} 
  \begin{xlist}
  \ex{\label{ex:datpl1}\gll mit [zwei Säcken [Äpfel]\Sub{Nom\slash Acc\slash Gen}]\Sub{Dat}\\
     with a sack apples\\
    \trans with a sack of apples}
    \ex{\label{ex:datpl2}\gll mit [zwei Säcken [Äpfeln]\Sub{Dat}]\Sub{Dat}\\
    with a sack apples\\}
  \end{xlist}
\end{exe}

Descriptive grammars seem to favour an analysis in terms of caselessness (for example, \citealp[1981]{ZifonunEa1997c}).
The hazy picture of the case of bare kind NPs is most likely due to the fact that case is so rarely marked on German nouns, where case is marked mostly on determiners and to some degree adjectives.
The uncertainty in the few cases where case can be marked (weak nouns in the singular and dative plurals) would thus be a direct consequence of the fact that the construction is not very specific with respect to the case of the kind noun.

\begin{table}
  \centering
  \begin{tabular}{llll}
    \multicolumn{1}{r}{kind NP is:} & bare noun NP & NP with adjective & NP with determiner \\
    & [\ldots N\Subsf{meas} [N\Subsf{kind}]] & [\ldots N\Subsf{meas} [AP N\Subsf{kind}]] & [\ldots N\Subsf{meas} [D N\Subsf{kind}]] \\
    \midrule
    \multirow{2}{*}{narrow apposition}
                & \NACb                                                 & \NACa                                                   & \multirow{2}{*}{---}       \\
		& (\ref{ex:intro:narrowapposition1}) \textit{Glas Wein} & (\ref{ex:intro:alternation2}) \textit{Glas guten Wein}  &                            \\
    \midrule

    \multirow{2}{*}{pseudo-partitive genitive} 
                & \multirow{2}{*}{---}                                  & \PGCa                                                   & \PGCd                      \\
                &                                                       & (\ref{ex:intro:alternation1}) \textit{Glas guten Weins} & (\ref{ex:intro:pseudopartitive1}) \textit{Glas dieses Weins} \\
  \end{tabular}
  \caption{NAC and PGC constructions in different NP structures with examples and references to full example sentences}
  \label{tab:constructions}
\end{table}

To summarise, the case patterns in the NAC and in the PGC (depending on the structure of the kind NP) is given in Table~\ref{tab:constructions}.
I call the narrow apposition construction with a bare kind noun the \NACb\ and the partitive genitive with a determiner in the kind NP the \PGCd.
For the alternants with an adjective but no determiner in the kind noun phrase I use the terms \NACa\ and \PGCa.
In principle, this paper is about the middle column of Table~\ref{tab:constructions}, \ie\ the syntactic configuration in which two different case patterns are acceptable.
However, in Section~\ref{sec:analyses}, the outer columns (\NACb\ and \PGCd) will still play a major role when the factors controlling the alternation are discussed.

\subsubsection{Neighbouring cases}
\label{sec:neighbouringcases}

% namely:
% 
% \begin{enumerate}[i.]
%   \item{\label{it:intro:idiot} alleged alternatives to the \NACa\ and the \PGCa,}
% %  \item{\label{it:intro:datsg} alternative \textit{weak} forms of dative singular neuter adjectives,}
% %  \item{\label{it:intro:femsg} case syncretism in feminine NPs,}
%   \item{\label{it:intro:plurl} similar constructions with plural/collective kind nouns,}
%   \item{\label{it:intro:noifl} grammaticalised non-inflected measure nouns, and}
%   \item{\label{it:intro:other} alternative constructions for expressing quantities.}
% \end{enumerate}
%
% \vspace{-1\baselineskip}

I now turn to some more subtle issues related to the measure noun case alternation in order to delimit the scope of the study.
First, there is a claim found in some grammars that a generic nominative, accusative, and even dative on the kind noun are used instead of the genitive (\PGCa) or case agreement (\NACa).
Overviews can be found in \cite{Hentschel1993} and \cite{Zimmer2015}.
Sentence (\ref{ex:intro:gennom}) shows a putative generic nominative on the kind noun inside an accusative MNP.

\begin{exe}
    \ex[*]{\label{ex:intro:gennom} Wir trinken [[eine Tasse]\Sub{Acc} [heißer Kaffee]\Sub{Nom}]\Sub{Acc}.}
\end{exe}

It was shown empirically in \cite{Hentschel1993} that such sentences are de facto not acceptable.
Also, in my corpus sample, they simply did not occur.
Even if they are accepted by some speakers, their extremely low frequency makes it virtually impossible to study them using corpus linguistic methods (Section~\ref{sec:corpusstudies}), and I consequently do not discuss them further.

% As for (\ref{it:intro:datsg}), a complication with neuter kind nouns in the dative is mentioned by \citet[20--22]{Zimmer2015}.
% In the \NACa, the adjective normally inflects with the so-called \textit{strong} case and number suffixes, which are used when no determiner with strong inflection is present in the NP.
% \cite{Zimmer2015} reports a high number of occurrences of adjectives being inflected with the \textit{weak} inflectional marker, which are normally only used if a strongly inflected determiner precedes the adjective: \textit{kalt-en Wasser} `cold water' instead of \textit{kalt-em Wasser}.%
% \footnote{See Section~\ref{sec:analyses} for more discussion of adjectival inflection.}
% In my corpus sample, this tendency was not nearly as clear, and there was a high number of very noisy sentences among those potentially showing this pattern.
% Hence, I do not discuss these forms.%
% \footnote{All corpus data and scripts used for this paper will be released freely, so further examination of the few cases maybe showing this kind of inflection is possible.}
% 
% \label{page:femininesyncretism}
% With feminine kind nouns, (\ref{it:intro:femsg}) is relevant.
% Whereas the singular masculine and neuter nominal subsystem is a four case system (at least for NPs with adjectives and determiners), feminine singular NPs show nominative-accusative and dative-genitive syncretisms.
% Table~\ref{tab:syncretisms} shows this as well as the partial syncretism in the plural, where no gender distinctions are made.
% This means that with dative measure nouns, the alternation is unobservable if the kind noun is feminine, and it is problematic to speak of a \textit{genitive} kind noun construction in this case.
% This is taken into account in the statistical modeling and the discussion presented in Section~\ref{sec:corpusstudies}.
% 
% \begin{table}
%   \centering
%   \begin{tabular}{llll}
%      & Masculine\slash Neuter & Feminine & Plural \\
%      \midrule
%      nominative & rot-er Wein    & \multirow{2}{*}{frisch-e Sahne}   & \multirow{2}{*}{klein-e Äpfel} \\
%      accusative & rot-en Wein    &                                   &                                \\
%      dative     & rot-em Wein    & \multirow{2}{*}{frisch-er Sahne}  & klein-en Äpfel-n               \\
%      genitive   & rot-en Wein-es &                                   & klein-er Äpfel                 \\
%   \end{tabular}
%   \caption{Case syncretisms in NPs with an adjective and without a determiner (\textit{roter Wein} `red wine', \textit{frische Sahne} `fresh cream', \textit{kleine Äpfel} `small apples')}
%   \label{tab:syncretisms}
% \end{table}

Second, we find that, if the kind noun is a plural count noun as in \textit{ein Sack kleine Äpfel} (\NACa) or \textit{ein Sack kleiner Äpfel} (\PGCa) `a bag of small apples', a similar alternation between PGC and NAC can be observed.
In line with experimental results reported in \citet[15--16]{Zimmer2015}, I found that the PGC is so dominant with plural kind nouns (794 of 861 cases, or over 92\%, cf.\ Section~\ref{sec:corpusstudies}) that the alternation cannot be analysed in the same way as in the singular.
%\footnote{More concretely, if these cases were included in the regression analysis of the corpus data along with a factor encoding the number of the kind noun, this factor would most assuredly override any other regressor for the data points with a plural kind noun.}
While this will play a role in the interpretation of the corpus findings, MNPs with plural kind nouns will not be included in the corpus study and the experiments reported in Sections~\ref{sec:corpusstudies} and \ref{sec:experimental}.

Third, some measure nouns have been grammaticalised in a way that they always appear in their non-inflected form.
They are typical measure nouns like \textit{Gramm} `gram', \textit{Pfund} `pound' or \textit{Prozent} `percent', which have no normal plural forms at all.%
\footnote{Plurals like \textit{Pfunde} and \textit{Prozente} have special meanings and have very restricted uses, mostly in idiomatic expressions such as \textit{Pfunde verlieren} `lose pounds' and \textit{Prozente machen} `make a profit'.
They cannot be used in normal MNPs.}
%\footnote{With neglectable frequency, this effect is extended to container nouns like \textit{Glas} `glass', but with a twist.
%A non-inflected version \textit{zwei Glas roter Wein} `two glasses of wine' and an inflected version \textit{zwei Gläser roter Wein} co-exist.
%The non-inflected version denotes the quantity of wine fitting into two glasses, the inflected version denotes two glasses filled with wine.}
I treat these cases like other measure nouns because they enter into both the \NACa\ as in (\ref{ex:uninflectedmeasures:a}) and the \PGCa\ as in (\ref{ex:uninflectedmeasures:b}).
In Section~\ref{sec:analyses}, however, degrees of grammaticalisation as a factor influencing the alternation will be discussed prominently.

\begin{exe}
  \ex\label{ex:uninflectedmeasures}
  \begin{xlist}
    \ex{\label{ex:uninflectedmeasures:a} zwei Gramm brauner Zucker}
    \ex{\label{ex:uninflectedmeasures:b}\gll zwei Gramm braunen Zuckers\\
                                         two gram brown sugar\\
				         two grams of brown sugar}
  \end{xlist}
\end{exe}

Finally, there are alternative ways of expressing similar quantificational meanings.
In the variationist tradition, which is strongly influenced by Labovian sociolinguistics, the \textit{principle of accountability} would dictate that proper studies should examine a variationist \textit{variable}, \ie\ all different \textit{ways of saying the same thing} (\citealp{Labov1966}, \citealp{Labov1969}, for an overview see \citealp{Tagliamonte2012}).
In the case at hand, the variable might be something like \textit{measurement of quantities of substances and collections}.
I argue that it is fully justified to focus narrowly on \NACa\ and \PGCa\ with their well-defined morpho-syntactic properties, mostly because the alternative constructions are not used in the same range of contexts.
Two major alternatives might be considered.
There is an analytic (pseudo-)partitive with \textit{von} `of'.
It is only available as an alternative to the PGC if the kind noun phrase contains a (definite or indefinite) determiner as in (\ref{ex:analyticalpartitive}).

\begin{exe}
  \ex\label{ex:analyticalpartitive} 
  \begin{xlist}
    \ex[ ]{\label{ex:analyticalpartitive:a}\gll ein Glas von dem roten Wein\\
                                                a glass of the red wine\\
                                                \glt a glass of the red wine
       }
    \ex[*]{\label{ex:analyticalpartitive:b}\gll ein Glas von rotem Wein\\
                                                a glass of red wine\\
                                                \glt a glass of red wine
       }
  \end{xlist}
\end{exe}

This means that the \textit{von} (pseudo-)partitive does not compete with the \NACa\ and the \PGCa.
In fact, there is not a single context in which they can be interchanged. 

Additionally, we find constructions with \textit{voll} (\textit{von})\slash\textit{voller} `full (of)' and \textit{mit} `with' are available, see (\ref{ex:alternatives}).

\begin{exe}
  \ex\label{ex:alternatives}
  \begin{xlist}
  \ex{\label{ex:voll}\gll   ein Glas voll von rotem Wein\\
                            a glass full of red wine\\
			    \glt a glass full of red wine}
  \ex{\label{ex:voller}\gll ein Glas {voll\slash voller} {rotem\slash roter} Wein\\
                            a glass full-of red wine\\
			    \glt a glass full of red wine}
  \ex{\label{ex:mit}\gll    ein Glas mit rotem Wein\\
                            a glass with red wine\\
			    \glt a glass with red wine in it\slash a glass filled with red wine}
  \end{xlist}
\end{exe}

These construction have very idiosyncratic properties.
The construction with \textit{voller} is discussed in \cite{Zeldes2018} and the construction with \textit{mit} in \cite{Bhatt1990}.
They are not semantically equivalent to the NAC and the PGC, and they are only available for a small subset of measure nouns.
All of these constructions are used only with measure nouns denoting containers, where the whole NP typically refers to the container and never to the quantity contained in it.
They are incompatible with measure nouns denoting natural portions such as \textit{Schluck} `gulp' or \textit{Haufen} `heap' and strongly grammaticalised nouns such as \textit{Gramm} `gram'.
This disqualifies them as alternatives to the \NACa\ or \PGCa\ in most contexts, and they are thus clearly not just more ways of saying the same thing. 
It is therefore reasonable to focus on the two well-defined alternants.

This concludes the descriptive overview of the phenomenon.
I have demonstrated that there is an alternation between two measure noun constructions in a narrow syntactic configuration (kind NP with an adjective but without a determiner), and that the two constructions differ in the case of the kind noun (case agreement with the measure noun or genitive).
I turn to more theory-oriented discussion of the alternation in the next section.


%%%%%%%%%%%%%%%%%%%%%%%%%%%%%%%%%%%%%%%%%%%%%%%%%%%%%%%%%%%%%%%%%%%%%%%
%%%%%%%%%%%%%%%%%%%%%%%%%%%%%%%%%%%%%%%%%%%%%%%%%%%%%%%%%%%%%%%%%%%%%%%

\subsection{Factors controlling the alternation}
\label{sec:analyses}

This section briefly reviews existing analyses of the \PGCa\ vs.\ \NACa\ alternation and related issues.
I also develop my own analysis and the appropriate hypotheses for the empirical studies presented in Sections~\ref{sec:corpusstudies} and \ref{sec:experimental}.




.%
\footnote{A reviewer mentioned that she or he had the impression that dialectal variation is also a factor influencing the alternation.
This is definitely true as some dialects (such as Alemannic) tend to have no genitive at all.
While this is an interesting aspect for further research, the main obstacle is that there are no corpora of German which are both large enough and annotated with reliable meta data about regional variants.
In the annotation of the corpus study, documents obviously written in a regional variant were excluded.
}

%%%%%%%%%%%%%%%%%%%%%%%%%%%%%%%%%%%%%%%%%%%%%%%%%%%%%%%%%%%%%%%%%%%%%%%
%%%%%%%%%%%%%%%%%%%%%%%%%%%%%%%%%%%%%%%%%%%%%%%%%%%%%%%%%%%%%%%%%%%%%%%
%%%%%%%%%%%%%%%%%%%%%%%%%%%%%%%%%%%%%%%%%%%%%%%%%%%%%%%%%%%%%%%%%%%%%%%


