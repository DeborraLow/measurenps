\section{Modelling the measure noun alternation}
\label{sec:germanmeasurenps}

In this section, I introduce and illustrate the relevant alternating constructions in Section~\ref{sec:descriptive}.
Then, I develop a model with prototype effects as well as exemplar effects based in existing research and my own theorising in Section~\ref{sec:analyses}.

\subsection{Alternating and non-alternating measure NP constructions}
\label{sec:descriptive}

I use the term \textit{measure noun phrase} (MNP) to refer to a noun phrase (NP) in which a kind-denoting (count or mass) noun depends on another noun that specifies a quantity of the objects or the substance denoted by the kind-denoting noun.
I call the kind-denoting noun the \textit{kind noun} and the quantity-denoting noun the \textit{measure noun}.
For illustration purposes, in the English phrase \textit{a glass of good wine}, the measure noun is \textit{glass} and the kind noun is \textit{wine}.
Measure nouns can be all sorts of nouns which denote a quantity (such as \textit{litre} or \textit{amount}) but also those denoting containers, collections, etc. (such as \textit{glass} or \textit{bucket}).
Like \citet[284]{Brems2003}, I also consider nouns ``which, strictly speaking, do not designate a `measure', but display a more nebulous potential for quantification'' to be measure nouns (also \citealp[530]{Koptjevskaja2001}, and \citealp[338]{Rutkowski2007}).

\subsubsection{The measure noun phrase alternation}
\label{sec:themeasurenounphrasealternation}

Three different syntactic configurations within MNPs need to be distinguished.
The case alternation only occurs when the kind noun is modified by an attributive adjective and there is no determiner, as in (\ref{ex:intro:alternation}).
Superficially, the sentences are functionally and semantically equivalent either with the kind noun in the genitive (\ref{ex:intro:alternation1}) or in the same case as the measure noun -- an accusative in the case of (\ref{ex:intro:alternation2}).%
\footnote{Some descriptive and normative grammars take stronger positions with regard to the acceptability of the two options.
See \cite{Hentschel1993} and \cite{Zimmer2015} for analyses of the sometimes absurd stances taken in grammars of German.
}
Notice that the genitive is virtually always used when the kind noun is a plural count noun, and the corresponding cases are consequently not included in the present study.

\begin{exe}
  \ex\label{ex:intro:alternation}
  \begin{xlist}
    \ex[ ]{\label{ex:intro:alternation1} \gll Wir trinken [[ein Glas]\Sub{Acc} [guten Weins]\Sub{Gen}]\Sub{Acc}.\\
    we drink a glass good wine \\
    \glt We drink a glass of good wine.}
    \ex[ ]{\label{ex:intro:alternation2} Wir trinken [[ein Glas]\Sub{Acc} [guten Wein]\Sub{Acc}]\Sub{Acc}.}
  \end{xlist}
\end{exe}

This specific configuration has to be seen in the context of two other configurations, to which I turn now.
First, if the kind noun forms an NP with a determiner, the construction resembles (and is usually called) a \textit{pseudo-partitive} (on partitives and pseudo-partitives see, e.\,g., \citealp{Barker1998,Selkirk1977,Stickney2007,Vos1999}; for a recent application of the terminology to German, see \citealp{Gerstenberger2015}).%
\footnote{If the kind noun is definite, the construction instantiates a true partitive.
Whereas partitives are constructions denoting a proper part-of relation as in \textit{a sip of the wine}, pseudo partitives -- albeit syntactically similar and diachronically related to partitives in many languages -- merely denote quantities and contain indefinite kind nouns as in \textit{a sip of wine}.
In the literature on German, some authors incorrectly call the pseudo-partitive a \textit{partitive} \citep{Hentschel1993} while some realise the difference and at least mention it \citep{Eschenbach1994,GallmannLindauer1994,Loebel1989,Zimmer2015}.
}
Here, the kind noun is in the genitive, and I refer to the construction in (\ref{ex:intro:pseudopartitive1}) as the \textit{Pseudo-partitive Genitive Construction} (PGC).

\begin{exe}
  \ex[ ]{\label{ex:intro:pseudopartitive1}\gll Wir trinken [[ein Glas]\Sub{Acc} [dieses Weins]\Sub{Gen}]\Sub{Acc}.\\
  we drink a glass this wine\\
  \glt We drink a glass of this wine.}
\end{exe}

Second, if the kind noun is bare -- \ie\ if it comes neither with a determiner nor a modifying adjective -- it is uninflected as in (\ref{ex:intro:narrowapposition1}), and the genitive as seen in the PGC is not acceptable, see (\ref{ex:intro:narrowapposition2}).%
\footnote{
It is difficult to determine whether the bare kind noun in (\ref{ex:intro:narrowapposition1}) bears no case at all, a generic case, or agrees in case with the measure noun.
When there is an adjective as in (\ref{ex:intro:alternation2}), the embedded kind NP clearly agrees in case, but due to the overall absence of markers of case in the singular, bare nouns mostly show no indication of their case.
Descriptive grammars seem to favour an analysis in terms of caselessness (for example, \citealp[1981]{ZifonunEa1997c}).
Since it is not relevant to the argument presented here, the question is left open. 
}

\begin{exe}
  \ex\label{ex:intro:narrowapposition}
  \begin{xlist}
    \ex[ ]{\label{ex:intro:narrowapposition1}\gll Wir trinken [[ein Glas]\Sub{Acc} [Wein]\Sub{?}]\Sub{Acc}.\\
    we drink a glass wine\\
    \glt We drink a glass of wine.}
    \ex[*]{\label{ex:intro:narrowapposition2} Wir trinken [[ein Glas]\Sub{Acc} [Weins]\Sub{Gen}]\Sub{Acc}.}
  \end{xlist}
\end{exe}

This construction is usually classified as a \textit{Narrow Apposition Construction} \citep{Loebel1986}, henceforth NAC.
Notice that the unavailability of the genitive on the kind noun follows independently from a constraint that genitive NPs in German require the presence of some strongly case-marked element (determiner or adjective) in addition to the head noun in order to be acceptable (\citealp{GallmannLindauer1994,Schachtl1989}; see also \citealp[160]{Eisenberg2013b}).

\begin{table}
  \centering
  \begin{tabular}{llll}
    \multicolumn{1}{r}{kind NP is:} & bare noun NP & NP with adjective & NP with determiner \\
    & [\ldots N\Subsf{meas} [N\Subsf{kind}]] & [\ldots N\Subsf{meas} [AP N\Subsf{kind}]] & [\ldots N\Subsf{meas} [D N\Subsf{kind}]] \\
    \midrule
    \multirow{2}{*}{narrow apposition}
                & \NACb                                                 & \NACa                                                   & \multirow{2}{*}{---}       \\
		& (\ref{ex:intro:narrowapposition1}) \textit{Glas Wein} & (\ref{ex:intro:alternation2}) \textit{Glas guten Wein}  &                            \\
    \midrule

    \multirow{2}{*}{pseudo-partitive genitive} 
                & \multirow{2}{*}{---}                                  & \PGCa                                                   & \PGCd                      \\
                &                                                       & (\ref{ex:intro:alternation1}) \textit{Glas guten Weins} & (\ref{ex:intro:pseudopartitive1}) \textit{Glas dieses Weins} \\
  \end{tabular}
  \caption{NAC and PGC constructions in different NP structures with examples and references to full example sentences}
  \label{tab:constructions}
\end{table}

To summarise, the case patterns in the NAC and in the PGC (depending on the structure of the kind NP) are given in Table~\ref{tab:constructions}.
I call the narrow apposition construction with a bare kind noun the \NACb\ and the partitive genitive with a determiner in the kind NP the \PGCd.
For the alternants with an adjective but no determiner in the kind noun phrase I use the terms \NACa\ and \PGCa.
In principle, this paper is about the middle column of Table~\ref{tab:constructions}, \ie\ the syntactic configuration in which two different case patterns are acceptable.
However, the outer columns (\NACb\ and \PGCd) will still play a major role when the factors controlling the alternation are discussed.


\subsubsection{Syntax of the alternating constructions}
\label{sec:syntax}

In order to understand the MNP alternation, it is vital to consider how in the relevant syntactic structure [N\Sub{1}~[A~N\Sub{2}]\Sub{NP\Sub{2}}]\Sub{NP\Sub{1}} (as instantiated in the \NACa\ and the \PGCa), the adjective is morpho-syntactically ambiguous between a determiner and a modifying adjective.
To show this, the so-called strong and weak inflection patterns of German adjectives needs to be taken into account.
In NPs with a strongly inflected determiner, attributive adjectives inflect according to the massively syncretic \textit{weak} pattern.
If there is no determiner (as is the case in the alternating constructions), attributive adjectives inflect like determiners themselves.
This is called the \textit{strong} inflectional pattern.
For example, in the dative (governed here by the preposition \textit{mit}), the strong suffix \textit{-em} is on the determiner in (\ref{ex:starkschwach:a}) and the adjective bears the weak suffix \textit{-en}. 
In (\ref{ex:starkschwach:b}), the strong suffix \textit{-em} appears on the adjective because there is no determiner.%
\footnote{In the masculine and neuter singular genitive, the strong and weak forms are indistinguishable.
Since the alternation itself does not occur in the genitive (see Section~\ref{sec:descriptive}), this does not create any complications for the present study.}

\begin{exe}
  \ex\label{ex:starkschwach} 
  \begin{xlist}
    \ex{\label{ex:starkschwach:a}\gll mit ein-em stark-en Kaffee\\
                                      with a strong coffee\\}
    \ex{\label{ex:starkschwach:b}\gll mit stark-em Kaffee\\
                                      with strong coffee\\}
  \end{xlist}
\end{exe}

Thus, the adjectives in the \NACa\ and the \PGCa\ have properties of adjectives as well as determiners.
On the one hand, they are lexical adjectives and function as attributive modifiers.
On the other hand, they are inflected like determiners, and they are the leftmost element in the NP, which is typical of determiners.
This unusual double nature of adjectives in NPs without determiners leads to a plausible interpretation of the pattern shown in Table~\ref{tab:constructions}.
If the adjective is classified as a determiner by virtue of carrying the inflectional markers which are otherwise characteristic of determiners, the \PGCa\ is the appropriate choice.
However, if it is classified as an adjective, the \NACa\ suggests itself because of the lack of a determiner.%
\footnote{The generative analysis presented in \cite{Bhatt1990} comes close to this interpretation by analysing the kind NP in the PGC as a DP and in the NAC as an NP.}
This morpho-syntactic ambiguity means that the \NACa\ is in fact a \NACb\ in disguise, and the \PGCa\ is a \PGCd\ in disguise.
This explains why the alternation arises in the first place.
In Section~\ref{sec:analyses}, I will argue that the NAC and the PGC constructions also have semantically distinguishable prototypes, and that the alternation between \PGCa\ and \NACa\ is an alternation between these prototypes.


\subsection{What controls the MNP alternation?}
\label{sec:analyses}

\subsubsection{Prototype effects}
\label{sec:prototypeeffects}

My analysis of what controls the MNP alternation is based first and foremost on the idea that the PGC prototype expresses a pseudo-partitive where two discernible entities -- the measure and the substance -- are referenced.
The NAC prototype, on the other hand, merely expresses a quantity, and the measure is not referenced as an entity in its own right.
Prototypical exemplars are given in (\ref{ex:prototypes1}) for the PGC and (\ref{ex:prototypes2}) for the NAC.

\begin{exe}
  \ex\label{ex:prototypes} 
  \begin{xlist}
    \ex{\label{ex:prototypes1}\gll Sie nahmen [einen Löffel [irgendeiner Medizin]\Sub{Gen}]\Sub{Acc}.\\
    they took a spoon some medication \\
    \trans They took a spoon(ful) of some medication.
  }
    \ex{\label{ex:prototypes2}\gll Sie kauften [drei Liter [Öl]\Sub{Acc\slash caseless}]\Sub{Acc}.\\
    they bought three liters oil \\
    \trans They bought three liters of oil.
  }
  \end{xlist}
\end{exe}

While the \PGCd\ in (\ref{ex:prototypes1}) allows an interpretation where speakers conceptualise the substance itself (\ie\ the medication) and an actual spoon used to measure the quantity of the medication, the \NACb\ in (\ref{ex:prototypes2}) does not allow such an interpretation.
While in the given examples, the effect is strongly supported by the choice of the measure noun lemmas, I argue that the meaning of the prototypes is independent of this.
The independent meanings of the prototypes are the result of diachronic developments and grammaticalisation processes.
Furthermore, I argue that the prototypical meanings are reflected in their usage patterns.

I begin by showing how the two meanings of the constructions emerge as a consequence of grammaticalisation processes of partitives and similar constructions.
It is often assumed that pseudo-partitives and quantity constructions arise as a form of grammaticalised partitives (\eg\ \citealp[536--539]{Koptjevskaja2001} for Finnish and Estonian, \citealp[559]{Koptjevskaja2001} for European languages in general).
The grammaticalisation path described by \citet[esp.\ 526--530]{Koptjevskaja2001} can start out (in some languages) with constructions involving two referential nouns (not even necessarily forming a single and contiguous NP) and a \textit{separative} meaning as in \textit{(cut) two slices from the cake} \citep[535]{Koptjevskaja2001}.
In this type of construction, it is most obvious that two separate referents (in the given example the cake and the slices) are conceptualised.
The \textit{part-of} meaning of true partitives as in \textit{a slice of the cake} represents the first stage of a development wherein the measure noun can already lose some semantic content, especially when words like \textit{bite} are no longer necessarily interpreted as a piece literally bitten out of something.
The pseudo-partitive stage finally instantiates a \textit{quantity-of} relation, potentially even leading to fully grammaticalised quantifiers such as \textit{a lot}.
In German, the two (now clearly distinct) available constructions have emerged diachronically from a single source through a complex reanalysis process, and the PGC is clearly the older construction \citep[2--4]{Zimmer2015}.
As predicted by the grammaticalisation pattern just described, it still has the potential to form a true partitive (if the kind noun is definite).
Conversely, the NAC lacks this ability to form true partitives and has gone further down the grammaticalisation path.
It is thus not surprising that it has lost (at least prototypically) the semantics which allows both the measure and the substance to be conceptually accessible as independent referents.
As a consequence, we should expect a tendency for the NAC constructions to host more strongly grammaticalised measure nouns.
For example, highly grammaticalised non-referential nouns like \textit{Gramm} `gram' and \textit{Meter} `metre' should occur proportionally more often in NAC constructions than in PGC constructions.
If such preferences can actually be shown in usage patterns, it would lend strong support for the hypothesised difference in the meanings of the prototypes.
In the corpus study, measure lemmas will therefore be annotated with appropriate semantic class labels to check whether semantic classes of measure nouns have different affinities to the two variants.
The actual classification used is based on the list in \citet[530]{Koptjevskaja2001}, but due to the low token frequencies in many of the potential classes, a very coarse classification was used in the end.
With typical examples, the classes are:
\textit{Physical} (typically non-referential precisely measurable units such as \textit{Liter} `litre', \textit{Meter} `metre', \textit{Gramm} `gram'),
\textit{Container} (\textit{Tasse} `cup'),
\textit{Amount} (\textit{Menge} `amount'), 
\textit{Portion} (natural portions like \textit{Happen} `bite' or \textit{Krümel} `crumb').
Notice that \textit{Container} is the class containing words like \textit{spoon} or \textit{cup}, which often develop into partially grammaticalised physical measure nouns.
The lemmas that did not fit into either of these classes were labelled \textit{Rest}.

A second preference should also be observable as a consequence of the different meanings.
As described above, the grammaticalisation path leads from NPs denoting individuated objects standing in a \textit{part-of} relation to a construction with a more diffuse \textit{quantity-of} relation.
Both types of relations can be numerically quantified -- inasmuch as a precise number of \textit{parts} or a numerically exact \textit{quantity} can be specified.
However, it is much more typical of quantities to be specified with numerical precision.
This is most obviously so with the highly grammaticalised physical measure nouns like \textit{centilitre}, which are very typically used with numerals instead of unspecific quantifiers, although both options are available in principle (\textit{three centilitres} vs.\ \textit{several centilitres}).
Since the \NACa\ is more closely associated with the \textit{quantity-of} relation, cardinals as attributes of the measure noun are expected to have a higher proportional frequency in the \NACa.
For illustration, (\ref{ex:determiners}) shows the expected alternants under this hypothesis.

\begin{exe}
  \ex\label{ex:determiners} 
  \begin{xlist}
    \ex{\label{ex:determiners:nac}\gll [[Drei Centiliter]\Sub{Nom} [heißer Rum]\Sub{Nom}]\Sub{Nom} sind genug.\\
                                       three centilitres hot rum are enough\\
				     \glt Three centilitres of hot rum is enough.}
    \ex{\label{ex:determiners:pgc}\gll [[Einige Centiliter]\Sub{Nom} [heißen Rums]\Sub{Gen}]\Sub{Nom} sind genug.\\
                                       some   centilitres hot    rum  are   enough\\
				     \glt A few centilitres of hot rum is enough.}
  \end{xlist}
\end{exe}

In (\ref{ex:determiners:nac}), the measure noun is modified by a cardinal \textit{drei} `three', and hence the \NACa\ is preferred.
In (\ref{ex:determiners:pgc}), the measure noun is modified by a non-cardinal determiner \textit{einige} `some', and the \PGCa\ is preferred.
Especially with exact physical measure nouns (like \textit{centilitre}), exact numerical quantification is invited.
By hypothesis, however, this goes beyond a selection effect tied to measure lemmas, and cardinal quantifiers are expected to co-occur relatively more often with the \NACa.

Finally, it was found that the \PGCa\ is more typical of higher stylistic levels (edited, closer to the non-regional standard, more formal) and\slash or even exclusive to written language (see \citealp[320--323]{Hentschel1993}).
The genitive -- an intrinsic part of the PGC -- is rarer in colloquial vernacular variants of German compared to the written standard.
This is the result of a diachronic process wherein some (but by no means all) uses of the genitive are being replaced by other cases or periphrastic constructions \citep{FleischerSchallert2011}.
Under an integral view of prototypes, which incorporates effects related to larger contexts and styles, such preferences can be part of what defines the construction prototypes, and the \PGCa\ should occur proportionally more often in more elaborate styles closer to the standard.

\subsubsection{Exemplar effects}
\label{sec:exemplareffects}

While the prototypes for the PGC and the NAC were specified with reference to high-level features in Section~\ref{sec:prototypeeffects}, at least one exemplar-driven similarity effect can also be expected to influence the MNP alternation.
The measure and kind noun lemmas which occur in the alternating constructions obviously also occur in the non-alternating constructions.
The relative frequency with which they occur in these stable and highly frequent cases -- where choosing an alternative is impossible -- could thus be a factor influencing the alternating, less stable case.
Should this be confirmed, it would be highly implausible to conceive of such an effect as a prototype effect.
In the corpus study reported in Section~\ref{sec:corpusstudies}, a measure quantifying this influence will therefore be included as the \textit{attraction strength}.

It should be noted that such an exemplar-type effect would have to be confirmed \textit{in addition} to the predicted semantic prototype effect for measure lemmas described in Section~\ref{sec:prototypeeffects}, namely the effect of semantic classes of measure nouns.
It must also be ensured that these influences at the lemma level are not spurious and just artefacts of mere lemma idiosyncrasies.
This leads to a rather complex and truly multilevel model structure for the generalised linear mixed models (GLMMs) to be used in Section~\ref{sec:corpusstudies}.%
\footnote{See \citet{GelmanHill2006} (especially part 2) for a comprehensive text book with a focus on multilevel models.}
While true multilevel models are not used very often in corpus linguistics -- \citet{Gries2015} even calls the simpler varying-intercepts and varying-slopes models ``underused'' -- they provide an excellent tool to describe situations where preferences at the sentence level and at lexical levels need to be integrated.
The models do not differentiate in and of themselves between abstraction and exemplar effects, but they allow researchers to tune the degree of complexity of models, incorporating both types of effects according to their analysis and the phenomenon at hand.
In Sections~\ref{sec:prestudy} and~\ref{sec:corpushierarchicalmodel}, multilevel models will therefore be used.
