\section{Modelling the measure noun alternation}
\label{sec:germanmeasurenps}

\subsection{Alternating and non-alternating measure NP constructions}
\label{sec:descriptive}

In this section, I introduce and illustrate the relevant alternating constructions.
I describe the narrowly defined syntactic configuration in which the alternation occurs, and I motivate the focus on \textit{only} this narrow range rather than, for example, the whole range of nominal constructions expressing quantities.

I use the term \textit{measure noun phrase} (MNP) to refer to a noun phrase (NP) in which a kind-denoting (count or mass) noun depends on another noun that specifies a quantity of the objects or the substance denoted by the kind-denoting noun.
I call the kind-denoting noun the \textit{kind noun} and the quantity-denoting noun the \textit{measure noun}.
For illustration purposes, in the English phrase \textit{a glass of good wine}, \textit{glass} is the measure noun and \textit{wine} is the kind noun.
Measure nouns can be all sorts of nouns which denote a quantity (such as \textit{litre} or \textit{amount}) but also those denoting containers, collections, etc. (such as \textit{glass} or \textit{bucket}).
Like \citet[284]{Brems2003}, I also consider nouns ``which, strictly speaking, do not designate a `measure', but display a more nebulous potential for quantification'' to be measure nouns (also \citealp[530]{Koptjevskaja2001}, and \citealp[338]{Rutkowski2007}).

\subsubsection{Core structures related to the alternation}

Three different syntactic configurations within MNPs need to be distinguished, and case alternation occurs only in one of them.
It occurs only when the kind noun is modified by an attributive adjective and there is no determiner, as in (\ref{ex:intro:alternation}).
Superficially, the sentences are functionally and semantically equivalent either with the kind noun in the genitive (\ref{ex:intro:alternation1}) or in the same case as the measure noun, an accusative in the case of (\ref{ex:intro:alternation2}).%
\footnote{Some descriptive and normative grammars take stronger positions with regard to the acceptability of the two options.
See \cite{Hentschel1993} and \cite{Zimmer2015} for analyses of the sometimes absurd stances taken in grammars of German.
As will be shown (especially in Section~\ref{sec:exp:fc}), there might be preferences, but we cannot assume either construction to be unacceptable.
}

\begin{exe}
  \ex\label{ex:intro:alternation}
  \begin{xlist}
    \ex[ ]{\label{ex:intro:alternation1} \gll Wir trinken [[ein Glas]\Sub{Acc} [guten Weins]\Sub{Gen}]\Sub{Acc}.\\
    we drink a glass good wine \\
    \glt We drink a glass of good wine.}
    \ex[ ]{\label{ex:intro:alternation2} Wir trinken [[ein Glas]\Sub{Acc} [guten Wein]\Sub{Acc}]\Sub{Acc}.}
  \end{xlist}
\end{exe}

This specific configuration has to be seen in the context of two other configurations, to which I turn now.
First, if the kind noun forms an NP with a determiner, the construction resembles (and is usually called) a \textit{pseudo-partitive} (on partitives and pseudo-partitives see, e.\,g., \citealp{Barker1998,Selkirk1977,Stickney2007,Vos1999}; for a recent application of the terminology to German, see \citealp{Gerstenberger2015}).%
\footnote{If the kind noun is definite, the construction instantiates a true partitive.
Whereas partitives are constructions denoting a proper part-of relation as in \textit{a sip of the wine}, pseudo partitives -- albeit syntactically similar and diachronically related to partitives in many languages -- merely denote quantities and contain indefinite kind nouns as in \textit{a sip of wine}.
In the literature on German, some authors incorrectly call the pseudo-partitive a \textit{partitive} \citep{Hentschel1993} while some realise the difference and at least mention it \citep{Eschenbach1994,GallmannLindauer1994,Loebel1989,Zimmer2015}.
}
Here, the kind noun is in the genitive, and I refer to the construction in (\ref{ex:intro:pseudopartitive1}) as the \textit{Pseudo-partitive Genitive Construction} (PGC).

\begin{exe}
  \ex[ ]{\label{ex:intro:pseudopartitive1}\gll Wir trinken [[ein Glas]\Sub{Acc} [dieses Weins]\Sub{Gen}]\Sub{Acc}.\\
  we drink a glass this wine\\
  \glt We drink a glass of this wine.}
\end{exe}

Second, if the kind noun is bare -- \ie\ if it comes neither with a determiner nor a modifying adjective -- it is uninflected as in (\ref{ex:intro:narrowapposition1}), and the genitive as seen in the PGC is not acceptable, see (\ref{ex:intro:narrowapposition2}).

\begin{exe}
  \ex\label{ex:intro:narrowapposition}
  \begin{xlist}
    \ex[ ]{\label{ex:intro:narrowapposition1}\gll Wir trinken [[ein Glas]\Sub{Acc} [Wein]\Sub{?}]\Sub{Acc}.\\
    we drink a glass wine\\
    \glt We drink a glass of wine.}
    \ex[*]{\label{ex:intro:narrowapposition2} Wir trinken [[ein Glas]\Sub{Acc} [Weins]\Sub{Gen}]\Sub{Acc}.}
  \end{xlist}
\end{exe}

This construction is usually classified as a \textit{Narrow Apposition Construction} \citep{Loebel1986}, henceforth NAC.%
\footnote{The construction as in (\ref{ex:intro:narrowapposition1}) is also referred to as the \textit{Direct Partitive Construction} for other Germanic languages in which the PGC with the synthetic genitive is not available.
This nomenclature makes sense in contrast to the \textit{Indirect Partitive Construction} with prepositional linkers translating to \textit{of} -- \ie\ analytic genitives -- in such languages, see \cite{HankamerMikkelsen2008} for Danish.
For German, this terminology is not distinctive enough, which is why I use the terms NAC and PGC.}
Notice that the unavailability of the genitive on the kind noun follows independently from a constraint that genitive NPs in German require the presence of some strongly case-marked element (determiner or adjective) in addition to the head noun in order to be acceptable (\citealp{GallmannLindauer1994,Schachtl1989}; see also \citealp[160]{Eisenberg2013b}).

It is difficult to determine whether the bare kind noun in the narrow apposition construction as in (\ref{ex:intro:narrowapposition1}) bears no case at all, a generic case, or agrees in case with the measure noun.
When there is an adjective as in (\ref{ex:intro:alternation2}), the embedded kind NP clearly agrees in case, but due to the overall absence of markers of case in the singular, bare nouns mostly show no indication of their case.
The only nouns which do have case markers in the singular are the so-called weak nouns \citep{Koepcke1995,Schaefer2016c}, which have something like a non-nominative \textit{-en} marker in the singular.
Unfortunately, there are no genuine mass nouns among the weak nouns.
However, a few of them can be coerced into a mass noun, such as \textit{Hase} `rabbit' (meaning `rabbit meat').
It then appears as if the uninflected form is preferred, but the inflected form is not excluded, at least for some speakers.
In (\ref{ex:macabre1}), the clearly acceptable form \textit{Hase} can only be a nominative singular or caseless.
In (\ref{ex:macabre2}), the form \textit{Hasen} could be an accusative, dative, or genitive.

\begin{exe}
  \ex\label{ex:macabre}
  \begin{xlist}
    \ex[ ]{\label{ex:macabre1}\gll Niemand will [ein Stück [Hase]\Sub{Nom\slash caseless}]\Sub{Acc} essen.\\
    nobody wants a piece rabbit eat\\
  \trans Nobody wants to eat a piece of rabbit.}
    \ex[?]{\label{ex:macabre2}\gll Niemand will [ein Stück [Hasen]\Sub{Acc\slash Dat\slash Gen}]\Sub{Acc} essen.\\
    nobody wants a piece rabbit eat\\}
  \end{xlist}
\end{exe}

Even a full unacceptability of (\ref{ex:macabre2}) would not be conclusive, however, as a possible aversion of speakers towards the case-marked form might be due to the fact that it is at least potentially also a genitive, in which case the constraint against bare genitive nouns would apply.
With plural kind nouns, the obligatory marking of all dative plurals with \textit{-en} (except those where this is phonotactically impossible) might provide some clues.
However, plural kind nouns do not behave like singular ones in measure phrases, as will be argued in Section~\ref{sec:neighbouringcases}.
Also, judgements vary between (\ref{ex:datpl1}) and (\ref{ex:datpl2}), and both are found in corpora.%
\footnote{For example, the variant in (\ref{ex:datpl1}) with the lemmas \textit{Sack} and \textit{Apfel} occurs four times, and the one in (\ref{ex:datpl2}) twice in the very large DECOW corpus (see Section~\ref{sec:gettingdata}).
Similarly, for [\textit{Kiste} [\textit{Äpfel}]\Sub{Nom\slash Acc\slash Gen}]\Sub{Dat} `box of apples' (no case identity), I find three examples, and [\textit{Kiste} [\textit{Äpfeln}]\Sub{Dat}]\Sub{Dat} (clearly marked case identity), I find six examples.
}

\begin{exe}
  \ex\label{ex:datpl} 
  \begin{xlist}
  \ex{\label{ex:datpl1}\gll mit [zwei Säcken [Äpfel]\Sub{Nom\slash Acc\slash Gen}]\Sub{Dat}\\
     with a sack apples\\
    \trans with a sack of apples}
    \ex{\label{ex:datpl2}\gll mit [zwei Säcken [Äpfeln]\Sub{Dat}]\Sub{Dat}\\
    with a sack apples\\}
  \end{xlist}
\end{exe}

Descriptive grammars seem to favour an analysis in terms of caselessness (for example, \citealp[1981]{ZifonunEa1997c}).
The hazy picture of the case of bare kind NPs is most likely due to the fact that case is so sparsely marked on nouns in contemporary German, where case is marked mostly on determiners and to some degree adjectives.
The uncertainty in the few cases where case can be marked (weak nouns in the singular and dative plurals as described above) would thus be a direct consequence of the fact that the construction is not very specific with respect to the case of the kind noun.

\begin{table}
  \centering
  \begin{tabular}{llll}
    \multicolumn{1}{r}{kind NP is:} & bare noun NP & NP with adjective & NP with determiner \\
    & [\ldots N\Subsf{meas} [N\Subsf{kind}]] & [\ldots N\Subsf{meas} [AP N\Subsf{kind}]] & [\ldots N\Subsf{meas} [D N\Subsf{kind}]] \\
    \midrule
    \multirow{2}{*}{narrow apposition}
                & \NACb                                                 & \NACa                                                   & \multirow{2}{*}{---}       \\
		& (\ref{ex:intro:narrowapposition1}) \textit{Glas Wein} & (\ref{ex:intro:alternation2}) \textit{Glas guten Wein}  &                            \\
    \midrule

    \multirow{2}{*}{pseudo-partitive genitive} 
                & \multirow{2}{*}{---}                                  & \PGCa                                                   & \PGCd                      \\
                &                                                       & (\ref{ex:intro:alternation1}) \textit{Glas guten Weins} & (\ref{ex:intro:pseudopartitive1}) \textit{Glas dieses Weins} \\
  \end{tabular}
  \caption{NAC and PGC constructions in different NP structures with examples and references to full example sentences}
  \label{tab:constructions}
\end{table}

To summarise, the case patterns in the NAC and in the PGC (depending on the structure of the kind NP) are given in Table~\ref{tab:constructions}.
I call the narrow apposition construction with a bare kind noun the \NACb\ and the partitive genitive with a determiner in the kind NP the \PGCd.
For the alternants with an adjective but no determiner in the kind noun phrase I use the terms \NACa\ and \PGCa.
In principle, this paper is about the middle column of Table~\ref{tab:constructions}, \ie\ the syntactic configuration in which two different case patterns are acceptable.
However, in Section~\ref{sec:analyses}, the outer columns (\NACb\ and \PGCd) will still play a major role when the factors controlling the alternation are discussed.


\subsubsection{A closer look at the syntax of the alternating constructions}
\label{sec:syntax}

In order to understand the MNP alternation, it is vital to consider how in the relevant syntactic structure [N\Sub{1}~[A~N\Sub{2}]\Sub{NP\Sub{2}}]\Sub{NP\Sub{1}} (as instantiated in the \NACa\ and the \PGCa), the adjective is morpho-syntactically ambiguous between a determiner and a modifying adjective.
To show this, the strong\slash weak inflection patterns of adjectives needs to be taken into account.
In NPs with a strongly inflected determiner, attributive adjectives inflect according to the massively syncretic \textit{weak} pattern.\footnote{The weak pattern has only two suffixes.
The structural cases (nominative and accusative) in the singular are marked with \textit{-e}, the oblique cases (dative and genitive) as well as the entire plural are marked with \textit{-en}.
The only exception to this rule is the masculine accusative singular, which is also marked with \textit{-en}.}
If there is no determiner (as is the case in the alternating constructions), attributive adjectives inflect like determiners themselves.
This is called the \textit{strong} inflectional pattern.
For example, in the dative (governed here by the preposition \textit{mit}), the strong suffix \textit{-em} is on the determiner in (\ref{ex:starkschwach:a}) and the adjective bears the weak suffix \textit{-en}. 
In (\ref{ex:starkschwach:b}), the strong suffix \textit{-em} appears on the adjective because there is no determiner.%
\footnote{In the masculine and neuter singular genitive, the strong and weak forms are indistinguishable.
Since the alternation itself does not occur in the genitive (see Section~\ref{sec:descriptive}), this does not create any complications for the present study.}

\begin{exe}
  \ex\label{ex:starkschwach} 
  \begin{xlist}
    \ex{\label{ex:starkschwach:a}\gll mit ein\textbf{-em} stark\textbf{-en} Kaffee\\
                                      with a strong coffee\\}
    \ex{\label{ex:starkschwach:b}\gll mit stark\textbf{-em} Kaffee\\
                                      with strong coffee\\}
  \end{xlist}
\end{exe}

Thus, the adjectives in the \NACa\ and the \PGCa\ have properties of adjectives as well as determiners.
On the one hand, they are lexical adjectives and function as attributive modifiers.
On the other hand, they are inflected like determiners, and they are the leftmost element in the NP, which is typical of determiners.
This unusual double nature of adjectives in NPs without determiners leads to a plausible interpretation of the pattern shown in Table~\ref{tab:constructions}.
Whenever speakers classify the adjective in the kind noun phrase more as a determiner, they use the \PGCa\ because, if there is a determiner, the PGC is the only option.
When they classify the adjective more as an adjective, the kind NP has no determiner, and they use the \NACa.%
\footnote{While the generative analysis presented in \cite{Bhatt1990} cannot properly deal with probabilistic effects, Bhatt comes close to this interpretation by analysing the kind NP in the PGC as a DP and in the NAC as an NP.}
This morpho-syntactic ambiguity means that the \NACa\ is in fact a \NACb\ in disguise, and the \PGCa\ is a \PGCd\ in disguise.
This explains why the alternation arises in the first place.
In Section~\ref{sec:analyses}, I will argue that the NAC and the PGC constructions also have semantically distinguishable prototypes, and that the alternation between \PGCa\ and \NACa\ is an alternation between these prototypes.

\subsubsection{Neighbouring cases}
\label{sec:neighbouringcases}

I now finally turn to some more subtle issues related to the measure noun case alternation in order to delimit the scope of the study.
First, there is a claim found in some grammars that a generic nominative, accusative, and even dative on the kind noun are used instead of the genitive (\PGCa) or case agreement (\NACa).
Overviews can be found in \cite{Hentschel1993} and \cite{Zimmer2015}.
Sentence (\ref{ex:intro:gennom}) shows a putative generic nominative on the kind noun inside an accusative MNP.

\begin{exe}
    \ex[*]{\label{ex:intro:gennom} Wir trinken [[eine Tasse]\Sub{Acc} [heißer Kaffee]\Sub{Nom}]\Sub{Acc}.}
\end{exe}

It was shown empirically in \cite{Hentschel1993} that such sentences are de facto not acceptable, and the results in \citet{Zimmer2015} are in line with her findings.
Also, in my corpus sample, they simply did not occur.
Even if they are accepted by some speakers, their extremely low frequency makes it virtually impossible to study them using corpus linguistic methods (Section~\ref{sec:corpusstudies}), and I consequently do not discuss them further.

Second, we find that, if the kind noun is a plural count noun as in \textit{ein Sack kleine Äpfel} (\NACa) or \textit{ein Sack kleiner Äpfel} (\PGCa) `a bag of small apples', a similar alternation between PGC and NAC can be observed.
In line with experimental results reported in \citet[15--16]{Zimmer2015}, I found that the PGC is so dominant with plural kind nouns (794 of 861 cases, or over 92\%, cf.\ Section~\ref{sec:corpusstudies}) that the alternation cannot be analysed in the same way as in the singular.
While this will play a role in the interpretation of the corpus findings, MNPs with plural kind nouns will not be included in the corpus study and the experiments reported in Sections~\ref{sec:corpusstudies} and \ref{sec:experimental}.

Third, some measure nouns have been grammaticalised in a way that they always appear in their non-inflected form.
They are typical measure nouns like \textit{Gramm} `gram', \textit{Pfund} `pound' or \textit{Prozent} `percent', which have no normal plural forms at all.%
\footnote{Plurals like \textit{Pfunde} and \textit{Prozente} have special meanings and have very restricted uses, mostly in idiomatic expressions such as \textit{Pfunde verlieren} `lose pounds' and \textit{Prozente machen} `make a profit'.
They cannot be used in normal MNPs.}
I treat these cases like other measure nouns because they enter into both the \NACa\ as in (\ref{ex:uninflectedmeasures:a}) and the \PGCa\ as in (\ref{ex:uninflectedmeasures:b}).
In Section~\ref{sec:analyses}, however, degrees of grammaticalisation as a factor influencing the alternation will be discussed prominently.

\begin{exe}
  \ex\label{ex:uninflectedmeasures}
  \begin{xlist}
    \ex{\label{ex:uninflectedmeasures:a} zwei Gramm brauner Zucker}
    \ex{\label{ex:uninflectedmeasures:b}\gll zwei Gramm braunen Zuckers\\
                                         two gram brown sugar\\
				         two grams of brown sugar}
  \end{xlist}
\end{exe}

Finally, there are alternative ways of expressing similar quantificational meanings.
In the variationist tradition, which is strongly influenced by Labovian sociolinguistics, the \textit{principle of accountability} would dictate that proper studies should examine a variationist \textit{variable}, \ie\ all different \textit{ways of saying the same thing} (\citealp{Labov1966}, \citealp{Labov1969}, for an overview see \citealp{Tagliamonte2012}).
In the case at hand, the variable might be something like \textit{measurement of quantities of substances and collections}.
I argue that it is fully justified to focus narrowly on \NACa\ and \PGCa\ with their well-defined morpho-syntactic properties, mostly because the alternative constructions are not used in the same range of contexts.
Two major alternatives might be considered.
There is an analytic (pseudo-)partitive with \textit{von} `of'.
It is only available as an alternative to the PGC if the kind noun phrase contains a (definite or indefinite) determiner as in (\ref{ex:analyticalpartitive}).

\begin{exe}
  \ex\label{ex:analyticalpartitive} 
  \begin{xlist}
    \ex[ ]{\label{ex:analyticalpartitive:a}\gll ein Glas von dem roten Wein\\
                                                a glass of the red wine\\
                                                \glt a glass of the red wine
       }
    \ex[*]{\label{ex:analyticalpartitive:b}\gll ein Glas von rotem Wein\\
                                                a glass of red wine\\
                                                \glt a glass of red wine
       }
  \end{xlist}
\end{exe}

This means that the \textit{von} (pseudo-)partitive does not compete with the \NACa\ and the \PGCa.
In fact, there is not a single context in which they can be interchanged. 

Additionally, we find constructions with \textit{voll} (\textit{von})\slash\textit{voller} `full (of)' and \textit{mit} `with' are available, see (\ref{ex:alternatives}).

\begin{exe}
  \ex\label{ex:alternatives}
  \begin{xlist}
  \ex{\label{ex:voll}\gll   ein Glas voll von rotem Wein\\
                            a glass full of red wine\\
			    \glt a glass full of red wine}
  \ex{\label{ex:voller}\gll ein Glas {voll\slash voller} {rotem\slash roter} Wein\\
                            a glass full-of red wine\\
			    \glt a glass full of red wine}
  \ex{\label{ex:mit}\gll    ein Glas mit rotem Wein\\
                            a glass with red wine\\
			    \glt a glass with red wine in it\slash a glass filled with red wine}
  \end{xlist}
\end{exe}

These construction have very idiosyncratic properties.
The construction with \textit{voller} is discussed in \cite{Zeldes2018} and the construction with \textit{mit} in \cite{Bhatt1990}.
They are not semantically equivalent to the NAC and the PGC, and they are only available for a small subset of measure nouns.
All of these constructions are used only with measure nouns denoting containers, where the whole NP typically refers to the container and never to the quantity contained in it.
They are incompatible with measure nouns denoting natural portions such as \textit{Schluck} `gulp' or \textit{Haufen} `heap' and strongly grammaticalised nouns such as \textit{Gramm} `gram'.
This disqualifies them as alternatives to the \NACa\ or \PGCa\ in most contexts, and they are thus clearly not just more ways of saying the same thing. 
It is therefore reasonable to focus on the two well-defined alternants.

This concludes the descriptive overview of the phenomenon.
I have demonstrated that there is an alternation between two measure noun constructions in a narrow syntactic configuration (kind NP with an adjective but without a determiner), and that the two constructions differ in the case of the kind noun (case agreement with the measure noun or genitive).
I turn to a more theory-oriented discussion of the alternation in the next section.


\subsection{What controls the MNP alternation?}
\label{sec:analyses}

In this section, I develop my analysis of the MNP alternation and the appropriate hypotheses for the empirical studies presented in Sections~\ref{sec:corpusstudies} and \ref{sec:experimental}.
I also review some existing analyses of the alternation and related issues.
It should be remembered from Section~\ref{sec:syntax} that, thanks to the syntactic ambiguity of strongly inflected adjectives, the \PGCa\ is plausibly a \PGCd\ in disguise and the \NACa\ a \NACb\ in disguise.

\subsubsection{Prototype effects}
\label{sec:prototypeeffects}

My analysis is based on the idea that the PGC prototype expresses a pseudo-partitive where two discernible entities -- the measure and the substance -- are referenced.
The NAC prototype, on the other hand, merely expresses a quantity, and the measure is not referenced as an entity in its own right.
Prototypical exemplars are given in (\ref{ex:prototypes1}) for the PGC and (\ref{ex:prototypes2}) for the NAC.

\begin{exe}
  \ex\label{ex:prototypes} 
  \begin{xlist}
    \ex{\label{ex:prototypes1}\gll Sie nahmen [einen Löffel [irgendeiner Medizin]\Sub{Gen}]\Sub{Acc}.\\
    they took a spoon some medication \\
    \trans They took a spoon(ful) of some medication.
  }
    \ex{\label{ex:prototypes2}\gll Sie kauften [drei Liter [Öl]\Sub{Acc\slash caseless}]\Sub{Acc}.\\
    they bought three liters oil \\
    \trans They bought three liters of oil.
  }
  \end{xlist}
\end{exe}

While the \PGCd\ in (\ref{ex:prototypes1}) allows an interpretation where speakers conceptualise the substance itself, \ie\ the medication, and a spoon used to take a quantity from the medication, the \NACb\ in (\ref{ex:prototypes2}) does not allow such an interpretation.
While in the given examples, the effect is strongly supported by the choice of the measure noun lemmas, I argue that the meaning of the prototypes is independent of this.%
\footnote{A reviewer asked for a specification of the \textit{schematic meaning} of the superordinate constructions in terms of \citet{Langacker1987}.
However, none of the constructions (\PGCd, \PGCa, \NACb, and \NACa) are exclusively associated with one of the discussed meanings, and thus I do not see how one could specify schemas (even at a superordinate level), for which ``membership is not a matter of degree'' and whose properties are ``fully compatible with all the members of the category'' \citep[371]{Langacker1987}.
A modelling in terms of a probabilistic similarity effects is more appropriate.
See also the argument in \citet[70--71]{Taylor2003} and my discussion in Section~\ref{sec:cogocl}.}
The independent meanings of the prototypes are, as I will show, a result of diachronic developments and grammaticalisation processes.
Furthermore, I argue that the prototypical meanings are reflected in their usage patterns, which will be tested in the corpus study in Section~\ref{sec:corpusstudies}.

I begin by showing how the two meanings of the constructions emerge as a consequence of grammaticalisation processes of partitives and similar constructions.
It is often assumed that pseudo-partitives and quantity constructions arise as a form of grammaticalised partitives (\eg\ \citealp[536--539]{Koptjevskaja2001} for Finnish and Estonian, \citealp[559]{Koptjevskaja2001} for European languages in general).
The grammaticalisation paths uncovered by \citet[esp.\ 526--530]{Koptjevskaja2001} are relevant for the case at hand.
The grammaticalisation path can start out (in some languages) with constructions involving two referential nouns (not even necessarily forming a single and contiguous NP) and a \textit{separative} meaning as in \textit{(cut) two slices from the cake} \citep[535]{Koptjevskaja2001}.
In this type of construction, it is most obvious that two separate referents (in the given example the cake and the slices) are conceptualised.
The \textit{part-of} meaning of true partitives as in \textit{a slice of the cake} represents the first stage of a development wherein the measure noun can already lose some semantic content, when, for example, words like \textit{bite} are no longer necessarily interpreted as a piece literally bitten out of something.
The pseudo-partitive stage finally instantiates a \textit{quantity-of} relation, potentially even leading to fully grammaticalised quantifiers such as \textit{a lot}.
In German, the PGC is clearly the older construction \citep{Zimmer2015}.
As predicted by the grammaticalisation pattern just described, it still has the potential to form a true partitive (if the kind noun is definite).
Conversely, the NAC lacks this ability to form true partitives and has gone further down the grammaticalisation path.
It is thus not surprising that it has lost (at least prototypically) the semantics which allows both the measure and the substance to be conceptually accessible as independent referents.
As a consequence, we should expect the NAC constructions to be typical hosting constructions for more strongly grammaticalised measure nouns.
For example, highly grammaticalised non-referential nouns like \textit{Gramm} `gram' and \textit{Meter} `metre' should occur proportionally more often in NAC constructions than in PGC constructions.
If such preferences can actually be shown in usage patterns, it would lend strong support for the hypothesised difference in the meanings of the prototypes.
In the corpus study, measure lemmas will therefore be annotated with appropriate semantic class labels to check whether semantic classes of measure nouns have different affinities to the two variants.
The actual classification is based on the list in \citet[530]{Koptjevskaja2001}, but due to the low frequencies of many of the potential classes, a very coarse classification was used in the end.
With typical examples, the classes are:
\textit{Physical} (abstract precisely measurable units such as \textit{Liter} `litre', \textit{Meter} `metre', \textit{Gramm} `gram'),
\textit{Container} (\textit{Eimer} `bucket'),
\textit{Amount} (\textit{Menge} `amount'), 
\textit{Portion} (natural portions like \textit{Happen} `bite' or \textit{Krümel} `crumb').
The lemmas that did not fit into either of these classes were labelled \textit{Rest}.


A second preference should also be observable as a consequence of the different meanings.
As described above, the grammaticalisation path leads from NPs denoting individuated objects standing in a \textit{part-of} relation to a construction with a more diffuse \textit{quantity-of} relation.
Both types of relations can be numerically quantified -- inasmuch as a precise number of \textit{parts} or a numerically exact \textit{quantity} can be specified.
However, it is much more typical of quantities to be specified with numerical precision.
This is most obviously so with the highly grammaticalised physical measure nouns like \textit{centilitre}, which are very typically used with exact numerals instead of unspecific quantifiers, although both options are available in principle (\textit{three centilitres} vs.\ \textit{several centilitres}).
Since the \NACa\ is more closely associated with the \textit{quantity-of} relation, cardinals as attributes of the measure noun are expected to have a higher proportional frequency in the \NACa.
For illustration, (\ref{ex:determiners}) shows the expected alternants under this hypothesis.

\begin{exe}
  \ex\label{ex:determiners} 
  \begin{xlist}
    \ex{\label{ex:determiners:nac}\gll [[Drei Centiliter]\Sub{Nom} [heißer Rum]\Sub{Nom}]\Sub{Nom} sind genug.\\
                                       three centilitres hot rum are enough\\
				     \glt Three centilitres of hot rum is enough.}
    \ex{\label{ex:determiners:pgc}\gll [[Einige Centiliter]\Sub{Nom} [heißen Rums]\Sub{Gen}]\Sub{Nom} sind genug.\\
                                       some   centilitres hot    rum  are   enough\\
				     \glt A few centilitres of hot rum is enough.}
  \end{xlist}
\end{exe}

In (\ref{ex:determiners:nac}), the measure noun is modified by a cardinal \textit{drei} `three', and hence the \NACa\ is preferred.
In (\ref{ex:determiners:pgc}), the measure noun is modified by a non-cardinal determiner \textit{einige} `some', and the \PGCa\ is preferred.
Especially with exact physical measure nouns (like \textit{centilitre}), exact numerical quantification is invited.
By hypothesis, however, this goes beyond a selection effect tied to measure lemmas, and cardinal quantifiers are expected to co-occur relatively more often with the \NACa.

Finally, it was found that the \PGCa\ is more typical of higher stylistic levels (distinctly edited, closer to the non-regional standard, more formal) and\slash or even exclusive to written language (see \citealp[320--323]{Hentschel1993}).
This is not surprising, given that the genitive -- an intrinsic part of the PGC -- is generally underrepresented in colloquial vernacular variants of German as a result of a diachronic process wherein many (but by no means all) uses of the genitive are being replaced by other cases or periphrastic constructions \citep{FleischerSchallert2011}.
Under an integral view of prototypes, which incorporates effects related to larger contexts and styles, such preferences can be part of what defines the construction prototypes, and the \PGCa\ should occur proportionally more often in more elaborate styles closer to the standard.%
\footnote{A reviewer mentioned that she or he had the impression that dialectal variation is also a factor influencing the alternation.
This is definitely true, as some dialects (such as Alemannic) tend to have no genitive at all.
While this is an interesting aspect for further research, the main obstacle for the present study is that there are no corpora of German which are both large enough and annotated with reliable metadata about regional variants.
In the annotation of the corpus study, documents obviously written in a regional variant were excluded.}

\subsubsection{Items, exemplar effects, and multilevel models}
\label{sec:itemandexemplareffects}

As discussed in Section~\ref{sec:cogocl}, prototype-based and exemplar-based approaches are merely the endpoints of a spectrum of theories allowing abstraction in cognitive representations to varying degrees.
While the prototypes for the PGC and the NAC were clearly specified with reference to abstract meaning in Section~\ref{sec:prototypeeffects}, at least one exemplar-driven similarity effect can also be expected to influence the MNP alternation, leading to a mixed model with abstractions as well as exemplar effects.
The measure and kind noun lemmas which occur in the alternating constructions obviously also occur in the non-alternating constructions.
The relative frequency with which they occur in these stable cases -- where choosing an alternative is impossible -- could thus be a factor influencing the alternating, less stable case.
Should this be confirmed, it would be highly implausible to conceive of such effects as prototype effects.
In the corpus study reported in Section~\ref{sec:corpusstudies}, a measure quantifying this influence will therefore be included as the \textit{attraction strength}.

It should be noted that such an exemplar-type effect would have to be confirmed \textit{in addition} to the predicted semantic prototype effect for measure lemmas described in Section~\ref{sec:prototypeeffects}, namely the effect of semantic classes of measure nouns.
This means that the statistical model needs to track an abstraction effect and an exemplar-like effect for measure nouns.
Additionally, controlling for simple lemma frequency effects is always a good idea, and it should be done for kind and measure nouns.
While, for example, very frequent kind nouns might preserve the older \PGCa, low-frequency kind nouns might tend to occur more often in the \NACa.
For measure nouns, high frequency might lead to a higher affinity to the \NACa\ if we assume at least a tendency for highly grammaticalised items to be more frequent.
Of course, the model must be specified in a way such that we can be sure that any detected frequency effect also goes beyond the semantic effect, which is captured in the semantic classes described in Section~\ref{sec:prototypeeffects}.
Such frequency effects would also be very difficult to describe as abstractions.

With all this, it must also be ensured that these influences at the lemma level (lemma type frequency, attraction strength, and semantic classes) are not spurious and just artefacts of mere lemma idiosyncrasies.
This leads to a rather complex and truly multilevel model structure for the generalised linear mixed models (GLMMs) customarily used in alternation modelling.%
\footnote{See \citet{Gries2015} for an argument in favour of varying-slope and varying-intercept models in corpus linguistic studies.
See \citet{GelmanHill2006} (especially part 2) for a comprehensive text book on the subject including multilevel models.}
A GLMM models the influence which several variables (\textit{predictors}) have on the probability that an alternant is chosen (the \textit{response}).
The so-called \textit{fixed effects} are assumed to be fixed population parameters which quantify the strength and the direction of the effect which, for example, a specific feature of the syntactic context or the stylistic level of the text have on the choice of the alternants.
The so-called \textit{random effects} are not fixed population parameters, but they vary by group.
The simplest random effect is a varying intercept, which predicts for groups of observations defined by lemmas, genres, speakers, etc. a constant term to be added to the model.
In a true multilevel model, however, the varying intercept itself comes with an additional linear model where second-level fixed effects predict the group-level effect, and this is exactly what is needed here.
To illustrate, the simple model specification in (\ref{eq:glmm}) represents the first level of a multilevel logistic regression model.

\begin{equation}
  Pr(y_i=1)=logit^{-1}(\beta^0_{j[i]}+\beta^1\cdot x^1_i)
  \label{eq:glmm}
\end{equation}

$Pr(y_1=1)$ is the predicted probability that the variable for the construction type in an observation (indexed by $i$) takes on the value 1 ($y_i=1$).
The varying intercept $\beta^0_{j[i]}$ adds a constant for group $j$ to which $i$ belongs (encoded here as $j[i]$, a notation borrowed from \citealp{GelmanHill2006}) to the linear term.%
\footnote{The linear term is the argument to the $logit^{-1}$ link function which transforms it into something which is interpretable as a probability.}
Any other $\beta$ is a fixed-effect coefficient, and each $x$ is an observation-level variable.
Here, just one fixed effect is included for illustration purposes, and $x^1$ could be, for example, the variable encoding whether a cardinal modifies the measure noun.
The observations (for example, the lines of a concordance) are indexed by $i$ and the groups\slash random intercepts by $j$.
Let us say the groups are defined by lemmas and we also include lemma frequency as a control variable.
In this case, a second-level model would be given similar to (\ref{eq:level2}).

\begin{equation}
  \beta^0_j\sim\mathcal{N}(\gamma_0+\gamma^1\cdot u_j, \sigma)
  \label{eq:level2}
\end{equation}

This says that the intercepts follow a normal distribution with a standard deviation of $\sigma$ and a mean predicted from the model $\gamma_0+\gamma^1\cdot u_j$, where $\gamma^0$ is the second-level intercept, the other $\gamma$ are the second-level coefficients, and each $u$ is a second-level predictor variable.
If $\beta^0_{j}$ were a lemma random effect, then $u_j$ would be the lemma frequency predictor which helps to predict the lemma intercept instead of just having a simple per-lemma constant.%
\footnote{Since the standard \textit{lme4} package takes care of multilevel modelling automatically, an R formula for (\ref{eq:glmm}) and (\ref{eq:level2}) could be \texttt{Construction\textasciitilde Cardinal+Lemmafrequency+(1|Lemma)} given that in the data set, the values of \texttt{Lemmafrequency} are unique for each \texttt{Lemma}.}

While multilevel models are rarely used in corpus linguistics -- \citet{Gries2015} even calls the simpler varying-intercepts and varying-slopes models ``underused'' -- they provide an excellent tool to describe situations where preferences at the sentence level and at lexical levels need to be integrated.
The models do not differentiate in and of themselves between abstraction and exemplar effects, but they allow researchers to tune the degree of complexity of models, incorporating both types of effects according to their theory and the phenomenon at hand.
In Section~\ref{sec:corpushierarchicalmodel}, a full multilevel model will therefore be used.
